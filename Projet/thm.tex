%%%%%%%%%%%%%%%%%%%%%%%%%%%%%%%%%%%%%%%%%%%%%%%%%%%%%%%%%%%%%%%%%%%%%%%%%%%%%%
%% STYLES  -- POUR LNCS

%\theoremstyle{plain} %style standard

% [David] Je commente tout ça...
% Pourquoi avoir theorm & co en plus des environnements theorem
% "normaux" (qu'on peut customiser de toute façon si on le souhaite
% vraiment) ?
%
%\newtheorem{theorm}{Theorem}[section]
%\newtheorem{propositin}[theorm]{Proposition}
%\newtheorem{definitin}[theorm]{Definition}
%\newtheorem{lema}[theorm]{Lemma}
%\newtheorem{corollay}[theorm]{Corollary}
%
%français
%\newtheorem{theorèe}[theorm]{Théorème}
%\newtheorem{définitin}[theorm]{Définition}
%\newtheorem{leme}[theorm]{Lemme}
%\newtheorem{corollaie}[theorm]{Corollaire}

%%%%%%%%%%%%%%%%%%%%%%%%%%%%%%%%%%%%%%%%%%%%%%%%%%%%%%%%%%%%%%%%%%%%%%%%%%%%%%
%% FONCTIONS
\def\upcase{\expandafter\makeupcase}
\def\makeupcase#1{\uppercase{#1}}

\newcommand{\optname}[1]{\ifthenelse{\equal{#1}{}}{}{\textbf{\upcase#1.}}}


%\newcommand{\preuve}[1]{\vspace{-5mm}\proof{#1}\qed}
%\newcommand{\preuve}[1]{}
\newcommand{\preuve}[1]{
\begin{proof}
#1 % \qed % Qed is part of proof env
\end{proof}}


%\newcommandx{\theorem}[2][1=]{
%\vbox{
%	\begin{theorm}\optname{#1}
%		#2
%	\end{theorm}
%	}
%}

%\newcommandx{\theoreme}[2][1=]{
%\vbox{
%	\begin{theorm}\optname{#1}
%		#2
%	\end{theorm}
%	}
%}


%\newcommandx{\proposition}[2][1=]{
%\vbox{
%	\begin{propositin}#1
%		#2
%	\end{propositin}
%	}
%}


%\newcommandx{\define}[2][1=]{
%\vbox{
%	\begin{definitin}\optname{#1}
%		#2
%	\end{definitin}
%	}
%}

%\newcommandx{\definition}[2][1=]{
%%\vbox{
%	\begin{définitin}\optname{#1}
%		#2
%	\end{définitin}
%%	}
%}


%\newcommandx{\lemma}[2][1=]{
%\vbox{
%	\begin{lema}\optname{#1}
%		#2
%	\end{lema}
%	}
%}

%\newcommandx{\lemme}[2][1=]{
%\vbox{
%	\begin{leme}\optname{#1}
%		#2
%	\end{leme}
%	}
%}


%\newcommandx{\corollary}[2][1=]{
%\vbox{
%	\begin{corollay}\optname{#1}
%		#2
%	\end{corollay}
%	}
%}

%\newcommandx{\corollaire}[2][1=]{
%\vbox{
%	\begin{corollaie}\optname{#1}
%		#2
%	\end{corollaie}
%	}
%}
