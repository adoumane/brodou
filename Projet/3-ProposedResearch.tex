\section{Proposed Research and its Context}

\subsection{Background}

\textbf{The Curry-Howard correspondence.}
Traditionally, proofs are a tool used to establish the validity
of mathematical statements. 
%
Proofs  have been generally  compared at an aestethic/concision level, but
were not considered as a proper object of study. The focus was in provability
 not in the proofs themselves.
%
This situation have changed in the 60's, where we descovered a deep link between proofs and programs. This link is known as the \textit{Curry-Howard Correspondence}.
%
\mytodo{Recall this correspondence.}
 

  Over time, the meaning of the Curry-Howard correspondence has evolved from
  the simple observation of an isomorphism between an existing
  proof system and an existing programming language, to something more “productive”.
Indeed, one can either start from a typed programming language and extract its isomorphic proof system which may have an interesting logical meaning.
Conversely, one could start from a known proof system and try to understand its computational content
in order to design its
  corresponding programming language,
  this is for instance how a number of calculi were  born~\cite{Parigot92,CurienH00, Munch09Foc}.

  Curry-Howard was also productive in renewing questions related to the
  semantics of logic, putting an emphasis on the semantics of proofs, in contrast to tranditional thruth or provability semantics. 
    
\bigskip 
  \textbf{Fixed points logics} are those logics containing two special connectives: the connective $\mu$ and its dual $\nu$. A formula of the form $\mu X. F$ (resp. $\nu X. F$) can be understood as the “least (resp. greatest) fixed points of the operator  $X\mapsto F(X)$”.
  
Least and greatest fixed points allow to treat in a direct, generic and intuitive way data (natural numbers, lists, etc) and co-data (streams, infinite trees, etc). The least fixed point operator $\mu$ allows to define
finite data types such as natural numbers,  lists, finite trees, etc.
Dually, the greatest fixed point operator allow to
define infinite data types (called co-data), such as streams, infinite trees, etc.


\bigskip
\textbf{Proof systems for logics with fixed points.}

There are two main families of proof systems with fixed points: finitary proof
systems with explicit rules of (co)induction and infinitary proof systems. 

\begin{itemize}
\item{Finitary proof systems.} In these proof systems, the induction  principle
is reflected by an  explicit rule, called \textbf{Park's rule}. Let us recall that the induction principle
is based on the characterization of the least fixed point of an operator $X\mapsto F(X)$, $\mu X. F$,
as its least pre-fixed point. This means that:
\begin{enumerate}
\item[i)] $\mu X. F$ is a pre-fixed point, \ie $F(\mu X. F)\leq \mu X. F$.
\item[ii)] $\mu X. F$ is smaller than any other pre-fixed point $S$ of $F$:
$$F(S)\leq S  \Rightarrow \mu X. F\leq S.$$
\end{enumerate}
These two points yield respectively the following two rules for the $\mu$ connective:
$$
\begin{array}{ll}
\begin{prooftree}
\Hypo{\Delta \vdash F[\mu X. F/X], \Gamma}
\Infer{1}[\rmur]{\Delta \vdash \mu X. F, \Gamma}
\end{prooftree}
&

  \begin{prooftree}
\Hypo{ F[S/X]\vdash S}
\Infer{1}[\rmul]{\mu X. F \vdash S}
\end{prooftree}

\end{array}
$$
Dually, the coinduction principle, based on the
characterization of the greatest fixed point of an operator as its
greatest post-fixed point, yields the following two rules:
$$
\begin{array}{cc}
\begin{prooftree}
\Hypo{S \vdash  F[S/X]}
\Infer{1}[\rnur]{S\vdash \nu X. F}
\end{prooftree}
&
\begin{prooftree}
\Hypo{ \Delta, F[\nu X. F/X] \vdash \Gamma}
\Infer{1}[\rnul]{\Delta, \nu X. F \vdash \Gamma}
\end{prooftree}
\end{array}
$$
%This way of treating fixed points in proof systems is the most popular.
The problem of these rules is that, in general, the cut rule is not admissible in
the proof systems using them. To recover cut-elimination, the following  variant of the rules  $\rmul$ and $\rnur$,
containing a  “hidden cut”, are usually used instead:
$$\begin{prooftree}
\Hypo{\Gamma\vdash S, \Delta}
\Hypo{ S \vdash F[S/X]}
\Infer{2}[\rnu]{\Gamma \vdash \nu X. F, \Delta}
\end{prooftree}
$$
$$
\begin{prooftree}
  \Hypo{F[S/X]\vdash S}
  \Hypo{\Gamma, S \vdash \Delta}
  \Infer{2}[\rmur]{\Gamma, \mu X. F \vdash\Delta}
\end{prooftree}
$$
\medskip
The fact that the cut rule is unavoidable (either in an explicit or a hidden way) means that proof systems with explicit rules of induction are fundamentally not suited
to proof-search, and that there is very little we can do to fix that. 
An other drawback of proofs using explicit induction  is that their computational meaning is usually
not explicit: it is hard to tell what
such proofs compute, when two proofs compute the same function, etc.

\item{Infinitary proof systems.} These systems are obtained by using, 
instead of the Park's rules $\rnur$ and $\rmul$, the following
unfolding rules:
$$\begin{prooftree}
\Hypo{ \Gamma\vdash F[\nu X. F/X], \Delta}
\Infer{1}[\rnur]{\Gamma\vdash \nu X. F, \Delta}
\end{prooftree}
\quad
\begin{prooftree}
\Hypo{ \Gamma, F[\mu X. F] \vdash \Delta}
\Infer{1}[\rmul]{\Gamma, \mu X. F \vdash \Delta}
\end{prooftree}
$$
and allowing infinite derivation trees, called \textbf{pre-proofs}.
Clearly, these rules do not reflect the difference in nature
between $\mu$ and $\nu$. More importantly, these pre-proofs are unsound, since one can derive the empty sequent as follows:
$$
\begin{prooftree}
\Hypo{\vdots}
\Infer{1}[\rmur]{\vdash \mu X. X}
\Infer{1}[\rmur]{\vdash \mu X. X}
\Hypo{\vdots}
\Infer{1}[\rmul]{ \mu X. X\vdash}
\Infer{1}[\rmul]{ \mu X. X\vdash}
\Infer{2}[\rcut]{\vdash }
\end{prooftree}
$$
To get a proof system which is sound, we declare a pre-proof to be a proof
if and only if it satisfies the \textbf{validity condition}. This condition says, roughly speaking, that in every infinite branch there should be either a least fixed point unfolded infinitely often in the left  or a greatest fixed unfolded infinitely often in the right.

There is a natural restriction to infinitary proofs given by regular proofs, that is, proofs that have only
finitely many sub-trees. These proofs are called \textbf{circular proofs}, since they can be represented
as finite trees with loops. This restricted \textbf{cyclic proof system} is more suited for a computer
science use, as its proofs can be finitely represented and manipulated.
\end{itemize}

\bigskip
Infinitary and circular  proof systems have existed for a long time, but in the shadows, and more as technical
tools than as proper proof systems. Indeed, their infinitary nature makes them suitable intermediary objects between the syntax and the semantics, so that we usually find them in
completeness proofs. This is the case, for instance, with the
\mucalculus \textbf{refutations} of Niwinski and Walukiewicz~\cite{NiwinskiW96}, which are used as an intermediary proof system in Walukiewicz's proof of completeness of the \mucalculus with respect to Kozen's
axiomatization~\cite{Waluk95}.

In the last decades, infinitary proof systems have started to come out from the shadows and have begun to be considered as proper proofs (Dam and Gurov~\cite{DamG02},  Dax \textit{et al.}~\cite{dhl06}. Santocanale ~\cite{santoCircularProofs}. Brotherston~\cite{}, QuodLibet~\cite{AvenhausKSW03}, $\dots$)

\medskip
Despite their compelling interest,
few proof-theoretical investigations have been done about infinitary proof theory (cut-elimination, focalization, $\dots$). 




\subsection{Research hypothesis and objectives}

\subsection{National importance}

\subsection{Programme and methodology}

\subsubsection{Task A: Circular functional programs}

\begin{itemize}
\item \textbf{Long term objective:} Design typed functional programming languages for (co)-inductive data-types based on circular type systems.

  \bigskip
The use of circular proofs as type systems represents a promising new
method of writing (co)inductive programs, at least from the three following perspectives:
  
\begin{enumerate}
\item The programs are easier to write, more user-friendly. Indeed, there are no invariants to guess.
\item It is known that every finitary proof can be transformed into a circular one, but the converse is still an open question (cf Task B).
Circular proofs may then contain ``more proofs'' and using circular type systems may allow us to write more programs.
  \item Another interesting feature of the infinitary proofs is that their computational meaning is more explicit compared to finitary proof systems.
  \mytodo{Add an example in the background section to argument that?}
\end{enumerate}

  
\item \textbf{Short/Mid-term objective:} Before realizing this long term objective, one needs to ensure that the circular type sytems
  enjoy the usual fundamental properties, normalization for instance.

  A step in this direction has been done by the second author and \textit{al.},  who showed cut-elimination for circular proofs in the setting of sequent calculus~\cite{}. It has been shown initially for multiplicative addivite linear-logic \MALL, but have been extended to other logics (\LL, \LK, \LJ).


  The translation between sequent calculus and
natural deduction is usually immediate in the finitary case. In the infinitary and the circular
setting, this is not the case anymore.


 Since circular typing systems  are usually in correspondence with the
formalism of natural deduction rather than with sequent calculus one, 
showing their normalization is not a corrolary of the second author's result.
Thus we have for this direction two short/mid-term objectives:
  \begin{enumerate}
  \item Define properly  a validity condition for natural deduction circular proofs.
    \item Show normalization.
    \end{enumerate}

  \end{itemize}

  \mytodo{We started working on that with David Baelde, Alexis Saurin and
  Guilhem Jaber, can we explicitely cite them as collaborators for this work direction?}
\subsubsection{Task B: The expressive power of circular proofs}


The question of relating the finitary proof systems with explicit (co)induction rules and the circular proof systems is a hard question.

It has been already set as a conjecture in similar settings, for instance
by Brotherston and Simpson~\cite{BrotherstonS07} in the framework of classical first-order logic with definitions.  


We can ask this question at two levels: at the level of \textbf{provability}
and at the level of \textbf{computational power}.  Before discussing these two work directions,
let us set up some notations.

\begin{notation}
  Let \Sys be a proof system for a logic \textit{without} fixed points.
  We denote by \muS the finitary proof system obtained from \Sys by adding
  the explicite (co)induction rules.

  We denote by \muSi the infinitary proof system obtained from \Sys by adding
  the unfolding rules for the fixed points and considering as proofs the
  infinite derivations satisfying the validity condition.

  We denote by \muSo the circular proof system obtained by restricting
  the proofs of \muSi to the regular ones. 
\end{notation}

\bigskip
For this task, we have two main research directions:
\begin{itemize}
\item \textbf{Relating the finitary and the circular proof systems at the level of provability:}
For a given proof system \Sys, does the proof systems \muS and \muSo prove the same theorems?

\medskip
In ~\cite{DoumanePhD}, this question has been partially answered by showing that every provable sequent in \muS is also provable in \muSo. The converse in more difficult, and ~\cite{DoumanePhD} provide only a sufficient condition on the
proof system \muS and \muSo (the invariant property) and a condition on the circular proofs to be translated (the translatability criterion) which guarantees that they can be indeed transformed into finitary proofs.

\bigskip
If the answer to this problem happens to be positive,  we see two ways to proceed:
\begin{itemize}
\item Either by finding a semantics of formulas, common to both \muS and \muSo. The idea is to show soundness for \muSo (that is, if a formula
  is provable in \muS then it is valid \wrt this semantics) and completeness for \muS (that is, if a formula is valid then it is provable in \muSo). This is the approach taken by Brotherstone and Simpson~\cite{BrotherstonS07}.
\item Or by using combinatorial techniques, that is by showing a translation procedure that transforms \muSo proofs into \muS ones. This is the approach adopted in \cite{DoumanePhD}. It has the advantage of being constructive by essence. 
\end{itemize}


Recently, the conjecture of Brotherston and Simpson has been 
 been answered by Berard and Tatsuta~\cite{BerardiT17} and by Alex Simpson~\cite{Simpson2017}.



\item \textbf{Relating the finitary and the circular proof systems at the level of computational power:}
For a given proof system \Sys, do the proof systems \muS and \muSo denote the same set of programs. 

If the answer happens to be positive, a way to proceed is by using \textbf{proof semantics}. 
For that, we have to find a domain of interpretation for proofs, common to both \muS and \muSo.
If we denote by $\interp{\pi}$, the interpretation of a \muS or a \muSo proof
$\pi$, this problem can be reformulated as follows:
For every proof $\Pi$ of \muSo (resp. \muS), is there a \muS (resp. \muSo) proof $\pi$ such that
$\interp{\Pi}=\interp{\pi}$?

An example of such proof semantics is computational ludics, which was used by the second author to compare the  \muMALLP and
\muMALLPo, the finitary and the circular extensions of polarized linear logic \MALLP.
\end{itemize}
