\documentclass[11pt,twocolumn]{article}

\usepackage[margin=2cm]{geometry}
\usepackage{multibib}
\newcites{track}{Selected publications by the team}
\newcites{biblio}{References}
\usepackage{gastex}
%\usepackage{prooftree}
\newcommand{\cyclist}{\textsc{Cyclist}}

\usepackage[all]{xy}
\usepackage{url}

%\usepackage{actes}
\usepackage{tikz}
\usetikzlibrary{arrows,automata,patterns}
\usepackage{pgf}
%%% extended tex
\usepackage{etex}

%%% Pour faire des joli todo
\usepackage{todonotes}

%%% encodage et langue
\usepackage[utf8]{inputenc}


\usepackage{hyphenat}

%%% mise en page
\usepackage{xcolor}
\usepackage{framed} %barres vertivales
\usepackage{blindtext} %lorem ipsum
\usepackage{multicol} %colonnes
\usepackage{caption}
  \DeclareCaptionType{copyrightbox}
\usepackage{subcaption}
\usepackage{wrapfig} %figures entourées de texte
 \usepackage{todonotes}
\presetkeys{todonotes}{fancyline, color=orange}{}
\usepackage{lscape}
\usepackage{rotating}
\usepackage{dsfont}
\usepackage{stmaryrd}
\newcommand{\para}[1]{\smallskip\noindent\emph{#1}.\xspace}

%%% symboles math, theoremes, police

\usepackage{amssymb,amsmath,amsthm}

\usepackage{thm-restate}


%%% figures
\usepackage{graphicx}
\usepackage{pifont}
\usepackage{qtree}

\usepackage{tikz}
\usetikzlibrary{trees}


\usepackage{ebproofs}

\usepackage{makecell}

\usepackage{bm}


%%%%%%%%%%%%%%%%%%%%%%%%%%%%%%%%%%%%%%%%%%%%%%%%%%%%%%%%%%%%%%%%%%%%%%%%%%%%%%
%% STYLES  -- POUR LNCS

%\theoremstyle{plain} %style standard

% [David] Je commente tout ça...
% Pourquoi avoir theorm & co en plus des environnements theorem
% "normaux" (qu'on peut customiser de toute façon si on le souhaite
% vraiment) ?
%
%\newtheorem{theorm}{Theorem}[section]
%\newtheorem{propositin}[theorm]{Proposition}
%\newtheorem{definitin}[theorm]{Definition}
%\newtheorem{lema}[theorm]{Lemma}
%\newtheorem{corollay}[theorm]{Corollary}
%
%français
%\newtheorem{theorèe}[theorm]{Théorème}
%\newtheorem{définitin}[theorm]{Définition}
%\newtheorem{leme}[theorm]{Lemme}
%\newtheorem{corollaie}[theorm]{Corollaire}

%%%%%%%%%%%%%%%%%%%%%%%%%%%%%%%%%%%%%%%%%%%%%%%%%%%%%%%%%%%%%%%%%%%%%%%%%%%%%%
%% FONCTIONS
\def\upcase{\expandafter\makeupcase}
\def\makeupcase#1{\uppercase{#1}}

\newcommand{\optname}[1]{\ifthenelse{\equal{#1}{}}{}{\textbf{\upcase#1.}}}


%\newcommand{\preuve}[1]{\vspace{-5mm}\proof{#1}\qed}
%\newcommand{\preuve}[1]{}
\newcommand{\preuve}[1]{
\begin{proof}
#1 % \qed % Qed is part of proof env
\end{proof}}


%\newcommandx{\theorem}[2][1=]{
%\vbox{
%	\begin{theorm}\optname{#1}
%		#2
%	\end{theorm}
%	}
%}

%\newcommandx{\theoreme}[2][1=]{
%\vbox{
%	\begin{theorm}\optname{#1}
%		#2
%	\end{theorm}
%	}
%}


%\newcommandx{\proposition}[2][1=]{
%\vbox{
%	\begin{propositin}#1
%		#2
%	\end{propositin}
%	}
%}


%\newcommandx{\define}[2][1=]{
%\vbox{
%	\begin{definitin}\optname{#1}
%		#2
%	\end{definitin}
%	}
%}

%\newcommandx{\definition}[2][1=]{
%%\vbox{
%	\begin{définitin}\optname{#1}
%		#2
%	\end{définitin}
%%	}
%}


%\newcommandx{\lemma}[2][1=]{
%\vbox{
%	\begin{lema}\optname{#1}
%		#2
%	\end{lema}
%	}
%}

%\newcommandx{\lemme}[2][1=]{
%\vbox{
%	\begin{leme}\optname{#1}
%		#2
%	\end{leme}
%	}
%}


%\newcommandx{\corollary}[2][1=]{
%\vbox{
%	\begin{corollay}\optname{#1}
%		#2
%	\end{corollay}
%	}
%}

%\newcommandx{\corollaire}[2][1=]{
%\vbox{
%	\begin{corollaie}\optname{#1}
%		#2
%	\end{corollaie}
%	}
%}


%%
% Pour les back-edges:
\newcommand{\point}[3]{
  \begin{tikzpicture}[remember picture,overlay]
    \node[inner sep=0pt,outer sep=0pt] (#2)  {\ensuremath{#3}};
    %\coordinate(R) at (0,0);
    \draw [->,>=latex] (#1.north) .. controls +(60:2cm) and +(-30:3cm) .. (#2.east);
  \end{tikzpicture}\;
}
\newcommand{\orig}[2]{\begin{tikzpicture}[remember picture]\node[inner sep=0pt,outer sep=1pt] (#1)  {\ensuremath{#2}};\end{tikzpicture}}
\newcommand{\destgauche}[3]{
  \begin{tikzpicture}[remember picture,overlay]
    \node[inner sep=0pt,outer sep=0pt] (#2)  {\ensuremath{#3}};
    %\coordinate(R) at (0,0);
    \draw [->,>=latex] (#1.north west) .. controls +(160:1.5cm) and +(200:1.5cm) .. (#2.west);
  \end{tikzpicture}\;
}
\newcommand{\destdroite}[3]{
  \begin{tikzpicture}[remember picture,overlay]
    \node[inner sep=0pt,outer sep=0pt] (#2)  {\ensuremath{#3}};
    %\coordinate(R) at (0,0);
    \draw [->,>=latex] (#1.north east) .. controls +(60:3cm) and +(-20:3cm) .. (#2.east);
  \end{tikzpicture}\;
}
%%

%%
% Affichage des threads
\makeatletter
\tikzset{%
  remember picture with id/.style={%
    remember picture,
    overlay,
    save picture id=#1,
  },
  save picture id/.code={%
    \edef\pgf@temp{#1}%
    \immediate\write\pgfutil@auxout{%
      \noexpand\savepointas{\pgf@temp}{\pgfpictureid}}%
  },
  if picture id/.code args={#1#2#3}{%
    \@ifundefined{save@pt@#1}{%
      \pgfkeysalso{#3}%
    }{
      \pgfkeysalso{#2}%
    }
  }
}

\def\savepointas#1#2{%
  \expandafter\gdef\csname save@pt@#1\endcsname{#2}%
}

\def\tmk@labeldef#1,#2\@nil{%
  \def\tmk@label{#1}%
  \def\tmk@def{#2}%
}

\tikzdeclarecoordinatesystem{pic}{%
  \pgfutil@in@,{#1}%
  \ifpgfutil@in@%
    \tmk@labeldef#1\@nil
  \else
    \tmk@labeldef#1,(0pt,0pt)\@nil
  \fi
  \@ifundefined{save@pt@\tmk@label}{%
    \tikz@scan@one@point\pgfutil@firstofone\tmk@def
  }{%
  \pgfsys@getposition{\csname save@pt@\tmk@label\endcsname}\save@orig@pic%
  \pgfsys@getposition{\pgfpictureid}\save@this@pic%
  \pgf@process{\pgfpointorigin\save@this@pic}%
  \pgf@xa=\pgf@x
  \pgf@ya=\pgf@y
  \pgf@process{\pgfpointorigin\save@orig@pic}%
  \advance\pgf@x by -\pgf@xa
  \advance\pgf@y by -\pgf@ya
  }%
}
\newcommand\tikzmark[2][]{%
\tikz[remember picture with id=#2] #1;}
\makeatother
%%


%%% back edges


%%



%%%%%%%%%%%%%%%%%%%%%%%%%%%%%%%%%
%%%%%%%%%%%%%%%%%%%%%%%%%%%%%%%%%
%                                                                                                                  %
%                                         MACROS                                                           %
%                                                                                                                  %
%%%%%%%%%%%%%%%%%%%%%%%%%%%%%%%%%
%%%%%%%%%%%%%%%%%%%%%%%%%%%%%%%%%

%%%%%%%%%%%%%%%%%%%%%%%%%%%%%%%%
%
%   ATTENTION, PAR CONVENTION AUCUN DES MACROS N'ENTRE NI
%                               NE SORT DU MATHMODE
%
%%%%%%%%%%%%%%%%%%%%%%%%%%%%%%%%


%%%%%%%%%%%%%%%%%%%%%%%%%%%%
%                                                                                                %
%                                     AUTRES                                              %
%                                                                                                %
%%%%%%%%%%%%%%%%%%%%%%%%%%%%












\newcommand{\defname}[1]{\textbf{\emph{#1}}}
\newcommand{\C}[1]{\mathcal{#1}}
\newcommand{\tr}{\mathsf{TR}}
\newcommand{\mapdef}[5]{\begin{tabular}{ccccl} $#1$ & $:$ & $#2$ & $\rightarrow$ & $#3$\\ & & $#4$ & $\mapsto$ & $#5$\\ \end{tabular}}
\newcommand{\powerset}[1]{\mathcal{P}(#1)}
\newcommand{\Pf}{\mathcal{P}_{fin}}
\def\pole{\perp\!\!\!\perp}

\newcommand\mytodo[1]{\textcolor{red}{[#1]}}
\newcommand{\semantics}[1]{\| #1 \|}

% \newcommand{\overview}{\noindent\textbf{Overview of the chapter.}}
% \newcommand{\readingadvices}{\noindent\textbf{Reading advices.}}
\newcommand{\ie}{\emph{i.e.},\xspace}
%\newcommand{\wlog}{\emph{w.l.o.g.},\xspace}
\newcommand{\comm}[1]{#1}
%\newcommand{\comm}[1]{}
\newcommand\scalemath[2]{\scalebox{#1}{\mbox{\ensuremath{\displaystyle #2}}}}
\newcommand{\rouge}[1]{\textcolor{red}{#1}}
\newcommand{\overview}{\paragraph*{Overview of the chapter.}}
\newcommand{\readingadvices}{\paragraph*{Reading advices.}}
\newcommand{\contrib}{\paragraph*{Contributions.}}
\newcommand{\addref}{\paragraph*{Additional references.}}
\newcommand{\termsignature}{\Sigma_{\lambda}}
%Indice à gauche
\newcommand{\indG}[2]{{\vphantom{#2}}_{#1}#2}

\newcommand{\successor}{succ}

\newcommand{\interaction}{\langle\mathcal{G}|\mathcal{A}\rangle}
\newcommand{\interp}[1]{\llbracket #1 \rrbracket}
\newcommand{\lfp}{\mathsf{lfp}}
\newcommand{\inv}{\ensuremath{\C{I}}}
\newcommand{\gfp}{\mathsf{gfp}}
\newcommand{\fv}{\mathsf{fv}}
\newcommand{\card}{\mathsf{card}}
\newcommand{\result}{\ensuremath{\mathsf{Result}}}
%%%%%%%%%%%%%%%%%%%%%%%%%%%%%
%                                                                                                   %
%                               ENVIRONNEMENTS                                      %
%                                                                                                   %
%%%%%%%%%%%%%%%%%%%%%%%%%%%%%



\newtheorem{theorem}{Theorem}
\newtheorem*{theorem*}{Theorem}
\newtheorem{proposition}{Proposition}
\newtheorem{lemma}{Lemma}
\newtheorem{corollary}{Corollary}
\newtheorem{observation}{Observation}
\newtheorem{conjecture}{Conjecture}
\theoremstyle{definition}
\newtheorem{remark}{Remark}
\newtheorem{example}{Example}
\newtheorem{definition}{Definition}
\newtheorem{terminology}{Terminology}
\newtheorem{notation}{Notation}
\newtheorem{proviso}{Proviso}
\newtheorem{convention}{Convention}

\renewcommand{\phi}{\varphi}
\renewcommand{\epsilon}{\varepsilon}
\renewcommand{\diamond}{\Diamond}




%%%%%%%%%%%%%%%%%%%%%%%%%%
% 
%                             Inférences
%
%%%%%%%%%%%%%%%%%%%%%%%%%%%

\newcommand{\reta}[1]{\scriptsize \ensuremath{(\eta_{\mathsf{#1}})}}
\newcommand\nat{\ensuremath{\mathsf{nat}}}
\newcommand\stream{\ensuremath{\mathsf{stream}}}
\newcommand\anyrule{\scriptsize \ensuremath{\mathsf{(r)}}}
\newcommand{\ruleocc}[2]{\scriptsize \ensuremath{\mathsf{(#1},#2\mathsf{)}}}
\newcommand\anyruleocc{\ruleocc{r}{x\rightarrow [[x^j]_j]_i}}
\newcommand\rax{\scriptsize \ensuremath{\mathsf{(Ax)}}}
\newcommand\rcut{\scriptsize \ensuremath{\mathsf{(Cut)}}}
\newcommand\rnext{\scriptsize \ensuremath{\mathsf{(\Next)}}}
\newcommand\rweak{\scriptsize \ensuremath{\mathsf{(W)}}}
\newcommand\rweakl{\scriptsize \ensuremath{\mathsf{(W_l)}}}
\newcommand\rweakr{\scriptsize \ensuremath{\mathsf{(W_r)}}}
\newcommand\rcont{\scriptsize \ensuremath{\mathsf{(C)}}}
\newcommand\rcontl{\scriptsize \ensuremath{\mathsf{(C_l)}}}
\newcommand\rcontr{\scriptsize \ensuremath{\mathsf{(C_r)}}}
\newcommand\rprom{\scriptsize \ensuremath{\mathsf{(!)}}}
\newcommand\rder{\scriptsize \ensuremath{\mathsf{(?)}}}
\newcommand\rex{\scriptsize \ensuremath{\mathsf{(Ex)}}}
\newcommand\rexl{\scriptsize \ensuremath{\mathsf{(Ex_l)}}}
\newcommand\rexr{\scriptsize \ensuremath{\mathsf{(Ex_r)}}}
\newcommand\rnegl{\scriptsize \ensuremath{\mathsf{(Neg_l)}}}
\newcommand\rnegr{\scriptsize \ensuremath{\mathsf{(Neg_r)}}}
\newcommand\rdisjl{\scriptsize \ensuremath{\mathsf{(\vee_{l})}}}
\newcommand\rdisjr{\scriptsize \ensuremath{\mathsf{(\vee_{r})}}}
\newcommand\rdisj{\scriptsize \ensuremath{\mathsf{(\vee)}}}
\newcommand\rbot{\scriptsize \ensuremath{\mathsf{(\bot)}}}
\newcommand\rbotr{\scriptsize \ensuremath{\mathsf{(\bot_r)}}}
\newcommand\rbotl{\scriptsize \ensuremath{\mathsf{(\bot_l)}}}
\newcommand\rconjl{\scriptsize \ensuremath{\mathsf{(\wedge_{l})}}}
\newcommand\rconjr{\scriptsize \ensuremath{\mathsf{(\wedge_{r})}}}
\newcommand\rconj{\scriptsize \ensuremath{\mathsf{(\wedge)}}}
\newcommand\rtop{\scriptsize \ensuremath{\mathsf{(\top)}}}
\newcommand\rtopl{\scriptsize \ensuremath{\mathsf{(\top_l)}}}
\newcommand\rtopr{\scriptsize \ensuremath{\mathsf{(\top_r)}}}
\newcommand\rsigmal{\scriptsize \ensuremath{\mathsf{(\sigma_{l})}}}
\newcommand\rsigmar{\scriptsize \ensuremath{\mathsf{(\sigma_{r})}}}
\newcommand\rsigma{\scriptsize \ensuremath{\mathsf{(\sigma)}}}
\newcommand\rmul{\scriptsize \ensuremath{\mathsf{(\mu_{l})}}}
\newcommand\rmur{\scriptsize \ensuremath{\mathsf{(\mu_{r})}}}
\newcommand\rnul{\scriptsize \ensuremath{\mathsf{(\nu_{l})}}}
\newcommand\rnur{\scriptsize \ensuremath{\mathsf{(\nu_{r})}}}
\newcommand\rmu{\scriptsize \ensuremath{\mathsf{(\mu)}}}
\newcommand\rnu{\scriptsize \ensuremath{\mathsf{(\nu)}}}
\newcommand\rnuU{\scriptsize \ensuremath{\mathsf{(\nu_{\mathsf{u}})}}}
\newcommand\functo[1]{\ensuremath{\mathsf{F_{#1}}}}
\newcommand\functoo[1]{\ensuremath{\mathsf{F}^\omega_{#1}}}
\newcommand\functoi[1]{\ensuremath{\mathsf{F}^\infty_{#1}}}
\newcommand{\rfuncto}[1]{\scriptsize \ensuremath{\mathsf{(F_{\mathsf{#1}})}}}
\newcommand{\rfunctoo}[1]{\scriptsize \ensuremath{\mathsf{(F^\omega_{\mathsf{#1}})}}}
\newcommand\rrightarrow{\scriptsize \ensuremath{\mathsf{(PwrSet)}}}
\newcommand\rs{\scriptsize \ensuremath{\mathsf{(S)}}}
\newcommand\rb{\scriptsize \ensuremath{\mathsf{(B)}}}
\newcommand\rassumption{\scriptsize \ensuremath{\mathsf{(A)}}}
\newcommand\ranys{\scriptsize \ensuremath{\mathsf{(s)}}}
\newcommand\runfoldr{\scriptsize \ensuremath{\mathsf{(Unfold_r)}}}
\newcommand\runfoldl{\scriptsize \ensuremath{\mathsf{(Unfold_l)}}}
\newcommand\rwith{\scriptsize \ensuremath{\mathsf{(\with)}}}
\newcommand\rparr{\scriptsize \ensuremath{\mathsf{(\parr)}}}
\newcommand\rotimes{\scriptsize \ensuremath{\mathsf{(\otimes)}}}
\newcommand\rtensor{\scriptsize \ensuremath{\mathsf{(\otimes)}}}
\newcommand\roplus{\scriptsize \ensuremath{\mathsf{(\oplus)}}}
\newcommand\roplusl{\scriptsize \ensuremath{\mathsf{(\oplus_1)}}}
\newcommand\roplusr{\scriptsize \ensuremath{\mathsf{(\oplus_2)}}}
\newcommand\roplusi{\scriptsize \ensuremath{\mathsf{(\oplus_i)}}}
\newcommand\roplusj{\scriptsize \ensuremath{\mathsf{(\oplus_j)}}}
\newcommand\rone{\scriptsize \ensuremath{\mathsf{(\llone)}}}
\newcommand\rmcut{\scriptsize \ensuremath{\mathsf{(mcut)}}}
\newcommand\rshiftdown{\scriptsize \ensuremath{\mathsf{(\shiftdown)}}}
\newcommand\rshiftup{\scriptsize \ensuremath{\mathsf{(\shiftup)}}}
\renewcommand{\phi}{\varphi}
\newcommand{\Ai}{i}
\newcommand{\Al}{l}
\newcommand{\Ar}{r}
%\newcommand\unfold{\ensuremath{\mathsf{Unfold}}}
\newcommand\unfold{\C{U}}
\newcommand\runfold{\scriptsize \ensuremath{\mathsf{(Unfold)}}}
\newcommand\runfoldI{\scriptsize \ensuremath{(\mathsf{Unfold}^\omega)}}
\newcommand\rReplacePhi{\scriptsize \ensuremath{\mathsf{(Subst)}}}
\newcommand\rReplacePhiI{\scriptsize \ensuremath{(\mathsf{Subst}^\omega)}}
\newcommand\rCloseGamma{\scriptsize \ensuremath{(\mathsf{Close})}}
\newcommand\rReplaceGamma{\scriptsize \ensuremath{(\mathsf{Replace})}}
\newcommand\runo{\scriptsize \ensuremath{\mathsf{(1)}}}
\newcommand\rdos{\scriptsize \ensuremath{\mathsf{(2)}}}
\newcommand\rtres{\scriptsize \ensuremath{\mathsf{(3)}}}
\newcommand\rquatro{\scriptsize \ensuremath{\mathsf{(4)}}}
%%%%%%%%%%%%%%%%%%%%%%%%%%%%%%
%
%                              Linear Logic
%
%%%%%%%%%%%%%%%%%%%%%%%%%%%%%%%%

\newcommand{\tensor}{{\otimes}}
\newcommand{\plus}{{\oplus}}
\newcommand{\llimp}{\multimap}
\newcommand{\llzero}{\mathbf{0}}
\newcommand{\llone}{\mathbf{1}}


%%%%%%%%%%%%%%%%%%%%%%%%%%%%%%%
%
%                          Proof sytem names
%
%%%%%%%%%%%%%%%%%%%%%%%%%%%%%%


\newcommand{\MALL}{\ensuremath{\mathsf{MALL}}\xspace}
\newcommand{\MALLP}{\ensuremath{\mathsf{MALLP}}\xspace}
\newcommand{\MALLso}{\ensuremath{\mathsf{MALL}2}\xspace}
\newcommand{\LK}{\ensuremath{\mathsf{LK}}\xspace}
\newcommand{\LTL}{\ensuremath{\mathsf{LTL}}\xspace}
\newcommand{\CTL}{\ensuremath{\mathsf{CTL}}}
\newcommand{\LKos}{\ensuremath{\mathsf{LK}_{\mathsf{os}}}\xspace}
\newcommand{\LKrev}{\ensuremath{\mathsf{LK}_{\mathsf{rev}}}\xspace}
\newcommand{\LL}{\ensuremath{\mathsf{LL}}\xspace}
\newcommand{\LKnext}{\ensuremath{\mathsf{LK}\odot}\xspace}

\newcommand{\muMALL}{\ensuremath{\mu\mathsf{MALL}}\xspace}
\newcommand{\muALL}{\ensuremath{\mu\mathsf{ALL}}\xspace}
\newcommand{\muMALLi}{\ensuremath{\mu\mathsf{MALL}^\infty}\xspace}
%\newcommand{\muMALLPi}{\ensuremath{\mu\mathsf{MALLP}^\infty}\xspace}
\newcommand{\muALLi}{\ensuremath{\mu\mathsf{ALL}^\infty}\xspace}
\newcommand{\muMALLo}{\ensuremath{\mu\mathsf{MALL}^\omega}\xspace}
\newcommand{\muALLo}{\ensuremath{\mu\mathsf{ALL}^\omega}\xspace}
\newcommand{\muMALLit}{\ensuremath{\mu\mathsf{MALL^\infty_\tau}}\xspace}
\newcommand{\muMALLim}{\ensuremath{\mu\mathsf{MALL}_\mathsf{m}^\infty}\xspace}
\newcommand{\muMALLP}{\ensuremath{\mu\mathsf{MALLP}}\xspace}
\newcommand{\muMALLPi}{\ensuremath{\mu\mathsf{MALLP}^\infty}\xspace}
\newcommand{\muMALLPo}{\ensuremath{\mu\mathsf{MALLP}^\omega}\xspace}
\newcommand{\muLL}{\ensuremath{\mu\mathsf{LL}}\xspace}
\newcommand{\muMLL}{\ensuremath{\mu\mathsf{MLL}}\xspace}
\newcommand{\muLLi}{\ensuremath{\mu\mathsf{LL}^\infty}\xspace}
\newcommand{\muLLo}{\ensuremath{\mu\mathsf{LL}^\omega}\xspace}


\newcommand{\muLK}{\ensuremath{\mu}\LK\xspace}
\newcommand{\mucLK}{\ensuremath{\mu_{\mathsf{c}}}\LK\xspace}
\newcommand{\mucMALL}{\ensuremath{\mu_{\mathsf{c}}}\MALL}
\newcommand{\muLKi}{\ensuremath{\mu\mathsf{LK}^\infty}\xspace}
\newcommand{\muLKo}{\ensuremath{\mu\mathsf{LK}^\omega}\xspace}
\newcommand{\Sstar}{\ensuremath{\mathsf{S}}\ensuremath{_{\mathsf{A}}}\xspace}
\newcommand{\muLKstar}{\ensuremath{\mu}\LK\ensuremath{_{\mathsf{A}}}\xspace}
\newcommand{\muMALLPA}{\ensuremath{\mu\mathsf{MALLP}_{\mathsf{A}}}\xspace}
\newcommand{\muMALLPoA}{\ensuremath{\mu\mathsf{MALLP}_{\mathsf{A}}^{\omega}}\xspace}
\newcommand{\muLKostar}{\ensuremath{\mu}\LK\ensuremath{_{\mathsf{A}}^{\omega}}\xspace}
\newcommand{\muLKh}{\ensuremath{\mu\mathsf{LK}^\omega_{\mathsf{DHL}}}\xspace}
\newcommand{\Sys}{\ensuremath{\mathsf{S}}\xspace}
\newcommand{\muS}{\ensuremath{\mu\mathsf{S}}\xspace}
\newcommand{\mucS}{\ensuremath{\mu_{\mathsf{c}}\mathsf{S}}\xspace}
\newcommand{\muSi}{\ensuremath{\mu\mathsf{S}^\infty}\xspace}
\newcommand{\muSo}{\ensuremath{\mu\mathsf{S}^\omega}\xspace}
\newcommand{\muSstar}{\ensuremath{\mu\mathsf{S}}\ensuremath{_{\mathsf{A}}}\xspace}
\newcommand{\muSostar}{\ensuremath{\mu\mathsf{S}}\ensuremath{_{\mathsf{A}}}\ensuremath{^{\omega}}\xspace}
\newcommand{\muLKnext}{\ensuremath{\mu\mathsf{LK}\odot}\xspace}
\newcommand{\muLKinext}{\ensuremath{\mu\mathsf{LK}\Next^\infty}\xspace}
\newcommand{\muLKonext}{\ensuremath{\mu\mathsf{LK}\odot^\omega}\xspace}



\newcommand{\encc}[1]{[#1]}
\newcommand{\nencc}[1]{\lfloor#1\rfloor}
%\newcommand{\Next}{\bigcirc}
\newcommand{\Next}{\odot}
\newcommand{\mucalculus}{$\mu$-calculus\xspace}
\newcommand{\malbdacalculus}{$\lambda$-calculus\xspace}
\newcommand{\At}{\mathcal{P}}

\newcommand\mcutred[1]{\underset{#1}{\longrightarrow}}

\long\def\hide#1\endhide{}
%% TODO et commentaires:

\newcommand{\comas}[1]{\todo
[inline,
linecolor=green!70!white, backgroundcolor=green!50!white, bordercolor=green
]
{Alexis: #1}}
\newcommand{\shortas}[1]{\todo
[linecolor=green!70!white, backgroundcolor=green!50!white, bordercolor=green]
{Alexis: #1}}

\newcommand{\comad}[1]{\todo
[
inline,
linecolor=cyan!70!white, 
backgroundcolor=cyan!50!white, bordercolor=cyan
]
{Amina: #1}}
\newcommand{\shortad}[1]{\todo
[linecolor=cyan!70!white, backgroundcolor=cyan!50!white, bordercolor=cyan]
{Amina: #1}}

\newcommand{\comdb}[1]{\todo[inline]{David: #1}}
\newcommand{\shortdb}[1]{\todo{David: #1}}

\newcommand{\comall}[1]{\todo[inline]{Commentaire: #1}}

\newcommand{\smalltodo}[1]{\todo{\footnotesize{#1}}}

% version courte/longue
%\DeclareOption{final}{\let\optshort\relax}
\DeclareOption{brouillon}{\let\optdraft\relax}
\DeclareOption{short}{\let\optshort\relax}
\DeclareOption{long}{\let\optlong\relax}

%\DeclareOption{short}{\@draftfalse\@shorttrue\@longfalse}
%\DeclareOption{long}{\@draftfalse\@shortfalse\@longtrue}

\ExecuteOptions{draft}
\ProcessOptions

%\pagestyle{plain}

\ifx\optdraft\undefined
\ifx\optshort\undefined
\ifx\optlong\undefined
\else
%%Version longue:
\newcommand{\boitecolor}[3]
    {\colorbox{#1}{\textcolor{#2}{#3}}}
\definecolor{gris}{rgb}{.45,.45,.45}
\definecolor{noir}{rgb}{0, 0, 0}
\long\def\beginlong#1\endlong{\boitecolor{gris}{noir}{#1}}
\long\def\beginlong#1\endlong{#1}
\long\def\beginshort#1\endshort{}
\fi
\else
%%Version courte:
\long\def\beginlong#1\endlong{}
\long\def\beginshort#1\endshort{#1}
\fi
\else
%%Version draft:
\long\def\beginlong#1\endlong{
\underline{\bf Pour version longue:}\\#1
~\\
\fbox{\bf Fin variante longue}\\
}
\long\def\beginshort#1\endshort{
\underline{\bf Pour version courte:}\\#1
~\\
\fbox{\bf Fin variante courte}\\
}
\fi


%%%%% Definitons en mode math

\newcommand{\fl}{\ensuremath{\mathsf{FL}}}
\newcommand{\mini}[1]{\mathsf{min}#1}
\newcommand{\infi}[1]{\mathsf{Inf}#1}
\newcommand{\Body}{\mathsf{Body}}

%%%%% ie st wrt ...
\newcommand{\eg}{\emph{e.g.}\xspace}
\newcommand{\st}{\emph{s.t.}\xspace}
\newcommand{\wrt}{\emph{w.r.t.}\xspace}


%%%%%%% Dashed lines
\newcommand\drawticks[1]{
\foreach \Valor in {-#1,...,#1}
  {
  \draw ([yshift=2pt]\Valor,0) -- ([yshift=-2pt]\Valor,0);  
  \draw ([xshift=2pt]0,\Valor) -- ([xshift=-2pt]0,\Valor);  
  }
}
\newcommand{\TT}[1]{\mathtt{#1}}
\newcommand{\dom}{\TT{Dom}}
\newcommand\rt{\mkrule{\tau}}
\newcommand{\mkrule}[1]{{\scriptsize\ensuremath{\mathsf{(#1)}}}\xspace}
\newcommand\rN{\mkrule{r_N}}
\newcommand{\ens}[1]{\{#1\}}

\newcommand{\SF}[1]{\mathsf{#1}}


% Shifts are \*arrows but not used as mathrels
\newcommand{\shiftdown}{{\downarrow}}
\newcommand{\shiftup}{{\uparrow}}
\newcommand{\lrule}[1]{\texttt{(#1)}}
\newcommand{\dai}{{\footnotesize\maltese}}
\newcommand{\arity}{\mathrm{ar}}
\newcommand{\daimon}{\textrm{\ding{64}}}

\newcommand{\ludicsmid}{\mid}
\newcommand{\PG}[4]{#1 \ludicsmid \overline{#2}\langle#3,\dots,#4\rangle}
\newcommand{\PB}[4]{#1 \ludicsmid \overline{#2}\langle #3,#4\rangle}
\newcommand{\PU}[3]{#1 \ludicsmid \overline{#2}\langle#3\rangle}
\newcommand{\PO}[2]{#1 \ludicsmid \overline{#2}}

\newcommand{\NEG}[4]{\sum #1(\vec{#2}_{#3}). #4}
\newcommand{\abot}{\bot}
\newcommand{\ashiftup}{\shiftup}
\newcommand{\awith}{\with}
\newcommand{\aparr}{\parr}
\newcommand{\aone}{1}
\newcommand{\ashiftdown}{\shiftdown}
\newcommand{\aplus}{\plus}
\newcommand{\atensor}{\tensor}

\newcommand{\dTenseur}[3]{#1 \ludicsmid {\atensor}\langle#2,#3\rangle}
\newcommand{\dOplus}[3]{#1 \ludicsmid \aplus_{#2}\langle#3\rangle}
\newcommand{\dDownShift}[2]{#1 \ludicsmid {\ashiftdown} \langle#2\rangle}
\newcommand{\dUn}[1]{#1 \ludicsmid \aone}

\newcommand{\dParr}[3]{ \aparr(#1,#2). #3}
\newcommand{\dWith}[4]{\awith_1(#1).#2 + \awith_2(#3).#4}
\newcommand{\dUpShift}[2]{{\ashiftup}(#1).#2}
\newcommand{\dTop}[1]{\sum a(\vec{#1}_a).\Omega}
\newcommand{\dBot}[1]{\abot.#1}
\newcommand{\NF}[1]{\llparenthesis #1\rrparenthesis}

\newcommand{\ra}{\rightarrow}
\makeatletter
\newcommand{\bigpole}{%
  \mathop{\mathpalette\bigp@rp\relax}%
  \displaylimits
}
\newcommand{\bigp@rp}[2]{%
  \vcenter{
    \m@th\hbox{\scalebox{\ifx#1\displaystyle2.1\else1.5\fi}{$#1\independent$}}
  }%
}
\newcommand\independent{\protect\mathpalette{\protect\independenT}{\perp}}
\def\independenT#1#2{\mathrel{\rlap{$#1#2$}\mkern2mu{#1#2}}}
\makeatother

\renewcommand{\bigpole}{\Perp}
\newcommand{\etc}{\emph{etc.}\xspace}
\newcommand{\dS}{\mathbf{S}}
\newcommand{\dC}{\mathbf{C}}


\def\etal{\textit{et al}\xspace}

%\def\At{\mathcal{P}}
%\newcommand{\fl}[1]{\mathsf{FL}(#1)}

%Safra construction
\newcommand{\fleche}[2]{{\overrightarrow{#1}}^{~#2}}
\newcommand{\transi}[1]{\stackrel{#1}{\rightarrow}} %Transition 
\newcommand{\safnode}[3]{(#1;#2)^#3} % Internal node: Trees, macrostates, adress
\newcommand{\safleaf}[2]{(#1)^#2}    % Unmarked leaf
\newcommand{\safmarked}[2]{[#1]^#2}  % Marked leaf
\newcommand{\lang}[1]{\mathcal{L}(#1)}  % langage recognized by an automaton (Büchi acceptance)
\newcommand{\langr}[1]{\mathcal{L}_r(#1)}  % langage recognized by an automaton (Rabin acceptance)
\newcommand{\Inf}[0]{\mathrm{Inf}}  % infinite support = \{q | \forall i, \exists j>i, \rho(j)=q\}
\newcommand{\myst}[0]{\mathrm{st}}  % macrostates of Safra's tree
\def\ceil#1{\lceil #1\rceil}        %all satets of a Safra tree
\def\P{\mathcal P}

\def\finstate{\mathcal F}
\def\Inf{\mathsf{Inf}}
\def\Run{\mathsf{Run}}
\def\Acc{\mathsf{Acc}}
\def\min{\mathsf{min}}
\def\max{\mathsf{max}}
%\def\ie{\textit{i.e.},\,}

\def\parityform{{\C F}_{{\vee}}}
\def\forget#1{\Phi({#1})}
%\def\forget#1{\overline{#1}}
\def\qh{\ensuremath{(Q,H)}}
\def\lv{\C{H}}




\begin{document}

\twocolumn[
\centerline{\Large A Curry-Howard Perspective On Cyclic Proofs: Case for Support} \vspace{0.3cm}
\centerline{James Brotherston (PI) and Amina Doumane (RCI)}
\centerline{Dept. of Computer Science, University College London}
\vspace{0.8cm}
]


\begin{abstract}
  \input{1-Abstract.tex}
\end{abstract}

\subsection*{1\quad Previous Research Track Record}

{\bf James Brotherston} is a Reader in Logic and Computation in the Dept.\ of Computing at University College London (UCL), where he is a member of the Programming Principles, Logic and Verification (PPLV) research group.  Brotherston holds a PhD from the School of Informatics, University of Edinburgh 2006.  He was a postdoctoral researcher at Imperial College London from 2006--11 before taking up a lectureship at Queen Mary University of London, moving to UCL in 2012. He has held as PI grants worth more than $\pounds$1M in total to date, including an EPSRC Postdoctoral Fellowship (2008--11) and an EPSRC Career Acceleration Fellowship (2011--16). He is the author of 29 conference and journal papers, with an $h$-index of $17$.

Brotherston's main research interests lie in mathematical logic and its application to computer science, particularly in automated program verification. He is especially well known for his pioneering work on \emph{cyclic proof} (see e.g.~\citetrack{BS11,BBC08,Bro05}) and also for his work in \emph{separation logic} (see e.g.~\citetrack{BK14,BV14}).  \\

\noindent {\bf Amina Doumane} is \ldots (INSERT BIO)

\paragraph{Relevant previous work.}
The investigators have contributed significantly to \ldots

\paragraph{Host organisation.}
The UCL Dept.\ of Computer Science is well known as a leading centre for computer science research in the UK, and was ranked first place of 89 universities for Computer Science in the 2014 REF. The PPLV group, headed by Prof.\ David Pym, consists of six full-time faculty members --- Pym, Brotherston, professors Alexandra Silva and Robin Hirsch, and lecturers Ilya Sergey and Fabio Zanasi --- plus a substantial number of postdocs, PhD students and part-time faculty. The group enjoys close links with Facebook (through Peter O'Hearn, part-time professor and Engineering Manager at Facebook) and Amazon (through Byron Cook, part-time professor and Head of AWS Security Automated Reasoning Group at Amazon) as well as the recently founded Turing Institute (through David Pym, ATI University Liaison Director for UCL) and numerous collaborative links with researchers at other universities.

\begingroup
\footnotesize
\bibliographystyletrack{alpha}
\bibliographytrack{track}
\endgroup

%\newpage

%\section{Previous Research Track Record}

\paragraph{James Brotherston}

\paragraph{Amina Doumane}

\paragraph{Host organisation}

\paragraph{Selected publications by the team}


\newpage

\section{Proposed Research and its Context}

\subsection{Background}

Good methods for reasoning with general inductive predicates have the potential to greatly boost the automation and coverage of such automatic verification tools.


However, the difficulty in reasoning with general inductive definitions --- a fundamental activity in both mathematics and computer science --- is well known, perhaps even notorious. Automating the process of mathematical reasoning using induction and recursion principles, generally known as \emph{inductive theorem proving}, has been a topic of active interest in computer science and AI research since at least the late 1970s, with the development of specialised theorem provers such as \textsc{NqThm}~\cite{Boyer-Moore:79}.
Since its inception, the default approach to automated inductive theorem proving has been \emph{explicit induction}: the direct application of proof rules or axioms capturing valid induction principles for the structures under consideration.  The best known explicit-induction provers in the last 20 years include ACL2~\cite{Kaufmann-Monolios-Moore:00}, \textsc{Inka}~\cite{Autexier-etal:99} and \textsc{Oyster-Clam}~\cite{Bundy-etal:90}, alongside more recent examples such as IsaPlanner~\cite{Dixon:PhD} and Zeno~\cite{Sonnex-etal:12}.  However, reasoning using explicit induction rules is known to be extremely difficult (for an overview see~\cite{Bundy:01}).  One of the biggest problems is that in an explicit induction proof, the structure(s) over which induction is to take place, the form of the induction schema and the induction hypotheses must all be chosen at the \emph{beginning} of the proof, before it is apparent which choices are likely to be fruitful.

An alternative to proof by explicit induction is provided by instead working in the \emph{infinite descent} style as proposed formally by Fermat~\cite{Mahoney:94} in which, roughly speaking, the use of explicit induction is eschewed in favour of a combination of simple \emph{case analysis} and a post-facto search for justification of the use of hypotheses by finding well-founded orderings.  Wirth~\cite{Wirth:Tech06} describes an imagined mathematician working in this way as follows:
\begin{quote}
{\em He starts with the conjecture and simplifies it by case analysis. When he realizes
that the current goal becomes similar to an instance of the conjecture, he applies
the instantiated conjecture just like a lemma, but keeps in mind that he has actually
applied an induction hypothesis. Finally, he searches for some well-founded ordering
in which all the instances of the conjecture he has applied as induction hypotheses
are smaller than the original conjecture.}%\flushright (Wirth~\cite{Wirth:Tech06})
\end{quote}
This approach has the advantage for proof search that the crucial choices of induction structures, schema and hypotheses are \emph{delayed until as late as possible in the proof}.



\emph{Cyclic proof systems} essentially formalise this concept: they drop explicit induction rules in favour of much simpler case analysis rules, but compensate by allowing for more complexity in the proof structure.  Specifically, a cyclic proof is a derivation tree whose leaves may be ``back-linked'' to another matching node in the proof (see Figure~\ref{fig:std_vs_cyclic_proof}), subject to a (decidable) global soundness condition ensuring that all infinite paths in this structure correspond to sound arguments by infinite descent.

%% JB NOTE: The following diagram is drawn using "prooftree", which is unfortunately not compatible with "ebproofs".

%\begin{figure*}[ht]
%\hspace{1.5cm}
%\begin{tabular}{cc}
%\begin{picture}(60,40)(0,0)
%{\small
%  \gasset{Nframe=n,Nadjust=wh,Nmr=0}
%  \node(root)(22,0){$\bullet$}
%  \node(inference)(30,13){$\begin{prooftree}\bullet \phantom{w} \bullet\, \cdots  \,\bullet \justifies \bullet \using \mbox{(Inference)}\end{prooftree}$}
%  \node(axiom1)(7,26){$\begin{prooftree}\justifies \phantom{w}\bullet\phantom{w} \using \mbox{(Axiom)}\end{prooftree}$}
%  \node(axiom2)(26.5,35){$\begin{prooftree}\justifies \phantom{w}\bullet\phantom{w} \using \mbox{(Axiom)}\end{prooftree}$}
%  \node(axiom3)(43,26){$\begin{prooftree}\justifies \phantom{w}\bullet\phantom{w}\using \mbox{(Axiom)}\end{prooftree}$}
%
%
%  \drawedge[dash={0.3 0.8}0,AHnb=0,exo=-8.2](root,inference){}
%  \drawedge[dash={0.3 0.8}0,AHnb=0,sxo=-8,exo=-12](inference,axiom1){}
%  \drawedge[dash={0.3 0.8}0,AHnb=0,sxo=-9.6,exo=-6.5](inference,axiom2){}
%  \drawedge[dash={0.3 0.8}0,AHnb=0,sxo=-4,exo=-6](inference,axiom3){}
%}
%\end{picture}
%&\hspace{0.8cm}
%\begin{picture}(60,40)(0,0)
%{\small
%  \gasset{Nframe=n,Nadjust=wh,Nmr=0}
%  \node(root)(22,0){$\bullet$}
%  \node(inference)(30,13){$\begin{prooftree}\bullet \phantom{w} \bullet\, \cdots  \,\bullet \justifies \bullet \using \mbox{(Inference)}\end{prooftree}$}
%  \node(axiom1)(6,26){$\begin{prooftree}\justifies \phantom{w}\bullet\phantom{w} \using \mbox{(Axiom)}\end{prooftree}$}
%  \node(bud1)(20,35){$\bullet$}
%  \node(bud2)(45,26){$\bullet$}
%
%
%  \drawedge[dash={0.3 0.8}0,AHnb=0,exo=-8.2](root,inference){}
%  \drawedge[dash={0.3 0.8}0,AHnb=0,sxo=-8,exo=-12](inference,axiom1){}
%  \drawedge[dash={0.3 0.8}0,AHnb=0,sxo=-9.6,exo=0](inference,bud1){}
%  \drawedge[dash={0.3 0.8}0,AHnb=0,sxo=-4,exo=0](inference,bud2){}
%  \drawedge[curvedepth=10,exo=-7](bud1,inference){}
%  \drawedge[curvedepth=16](bud2,root){}
%}
%\end{picture}
%\end{tabular}
%\caption{\small Left: a typical proof structured as a finite tree, with the parent-child relation between nodes ($\bullet$) given by a set of inference rules. Right: a typical \emph{cyclic pre-proof}, structured as a tree proof with ``back-links'' between nodes (shown as arrows).\label{fig:std_vs_cyclic_proof}}%\vspace{-0.5cm}
%\end{figure*}

Cyclic proof systems seem to have first arisen in computer science as tableaux for the propositional modal $\mu$-calculus~\cite{Stirling-Walker:91}.  Since then, cyclic proof systems have been proposed for a number of applications, including: first-order $\mu$-calculus~\cite{Dam-Gurov:02};
%Sprenger-Dam:03}
verifying properties of concurrent processes~\cite{Schopp-Simpson:02}; first-order logic with inductive definitions~\cite{Bro05}, bunched logic~\cite{Brotherston:07}; and termination of pointer programs~\cite{BBC08}.  For some time, the development of these formal proof systems was not matched by the development of automated tools implementing them, most likely owing to the technical difficulties of dealing with cyclic proof structures subject to a global soundness condition.  However, in the last few years a small number of tools based on cyclic proof have appeared, including the interactive \textsc{QuodLibet} prover~\cite{Wirth-etal:03}, and provers geared to limited fragments of separation logic~\cite{Nguyen-Chin:08,Brotherston-Distefano-Petersen:11}. Most recently, a prototype theorem prover, $\cyclist$, has been developed that implements both the generic notion of cyclic proof due to Brotherston and its instantiation to many of the aforementioned concrete settings, including the abduction of inductive predicates for program termination~\cite{BGP12,BG12}.  This prover exhibits promising automated performance on examples that ordinarily require complicated induction schemes, which are problematic for traditional explicit-induction theorem provers.

\textbf{The Curry-Howard correspondence.}
Traditionally, proofs are a tool used to establish the validity
of mathematical statements.
%
Proofs  have been generally  compared at an aestethic/concision level, but
were not considered as a proper object of study. The focus was in provability
 not in the proofs themselves.
%
This situation have changed in the 60's, where we descovered a deep link between proofs and programs. This link is known as the \textit{Curry-Howard Correspondence}.
%
\mytodo{Recall this correspondence.}


  Over time, the meaning of the Curry-Howard correspondence has evolved from
  the simple observation of an isomorphism between an existing
  proof system and an existing programming language, to something more “productive”.
Indeed, one can either start from a typed programming language and extract its isomorphic proof system which may have an interesting logical meaning.
Conversely, one could start from a known proof system and try to understand its computational content
in order to design its
  corresponding programming language,
  this is for instance how a number of calculi were  born~\citebiblio{Parigot92,CurienH00, Munch09Foc}.

  Curry-Howard was also productive in renewing questions related to the
  semantics of logic, putting an emphasis on the semantics of proofs, in contrast to tranditional thruth or provability semantics.

\bigskip
  \textbf{Fixed points logics} are those logics containing two special connectives: the connective $\mu$ and its dual $\nu$. A formula of the form $\mu X. F$ (resp. $\nu X. F$) can be understood as the “least (resp. greatest) fixed points of the operator  $X\mapsto F(X)$”.

Least and greatest fixed points allow to treat in a direct, generic and intuitive way data (natural numbers, lists, etc) and co-data (streams, infinite trees, etc). The least fixed point operator $\mu$ allows to define
finite data types such as natural numbers,  lists, finite trees, etc.
Dually, the greatest fixed point operator allow to
define infinite data types (called co-data), such as streams, infinite trees, etc.


\bigskip
\textbf{Proof systems for logics with fixed points.}

There are two main families of proof systems with fixed points: finitary proof
systems with explicit rules of (co)induction and infinitary proof systems.

\begin{itemize}
\item{Finitary proof systems.} In these proof systems, the induction  principle
is reflected by an  explicit rule, called \textbf{Park's rule}. Let us recall that the induction principle
is based on the characterization of the least fixed point of an operator $X\mapsto F(X)$, $\mu X. F$,
as its least pre-fixed point. This means that:
\begin{enumerate}
\item[i)] $\mu X. F$ is a pre-fixed point, \ie $F(\mu X. F)\leq \mu X. F$.
\item[ii)] $\mu X. F$ is smaller than any other pre-fixed point $S$ of $F$:
$$F(S)\leq S  \Rightarrow \mu X. F\leq S.$$
\end{enumerate}
These two points yield respectively the following two rules for the $\mu$ connective:
$$
\begin{array}{ll}
\begin{prooftree}
\Hypo{\Delta \vdash F[\mu X. F/X], \Gamma}
\Infer{1}[\rmur]{\Delta \vdash \mu X. F, \Gamma}
\end{prooftree}
&

  \begin{prooftree}
\Hypo{ F[S/X]\vdash S}
\Infer{1}[\rmul]{\mu X. F \vdash S}
\end{prooftree}

\end{array}
$$
Dually, the coinduction principle, based on the
characterization of the greatest fixed point of an operator as its
greatest post-fixed point, yields the following two rules:
$$
\begin{array}{cc}
\begin{prooftree}
\Hypo{S \vdash  F[S/X]}
\Infer{1}[\rnur]{S\vdash \nu X. F}
\end{prooftree}
&
\begin{prooftree}
\Hypo{ \Delta, F[\nu X. F/X] \vdash \Gamma}
\Infer{1}[\rnul]{\Delta, \nu X. F \vdash \Gamma}
\end{prooftree}
\end{array}
$$
%This way of treating fixed points in proof systems is the most popular.
The problem of these rules is that, in general, the cut rule is not admissible in
the proof systems using them. To recover cut-elimination, the following  variant of the rules  $\rmul$ and $\rnur$,
containing a  “hidden cut”, are usually used instead:
$$\begin{prooftree}
\Hypo{\Gamma\vdash S, \Delta}
\Hypo{ S \vdash F[S/X]}
\Infer{2}[\rnu]{\Gamma \vdash \nu X. F, \Delta}
\end{prooftree}
$$
$$
\begin{prooftree}
  \Hypo{F[S/X]\vdash S}
  \Hypo{\Gamma, S \vdash \Delta}
  \Infer{2}[\rmur]{\Gamma, \mu X. F \vdash\Delta}
\end{prooftree}
$$
\medskip
The fact that the cut rule is unavoidable (either in an explicit or a hidden way) means that proof systems with explicit rules of induction are fundamentally not suited
to proof-search, and that there is very little we can do to fix that.
An other drawback of proofs using explicit induction  is that their computational meaning is usually
not explicit: it is hard to tell what
such proofs compute, when two proofs compute the same function, etc.

\item{Infinitary proof systems.} These systems are obtained by using,
instead of the Park's rules $\rnur$ and $\rmul$, the following
unfolding rules:
$$\begin{prooftree}
\Hypo{ \Gamma\vdash F[\nu X. F/X], \Delta}
\Infer{1}[\rnur]{\Gamma\vdash \nu X. F, \Delta}
\end{prooftree}
\quad
\begin{prooftree}
\Hypo{ \Gamma, F[\mu X. F] \vdash \Delta}
\Infer{1}[\rmul]{\Gamma, \mu X. F \vdash \Delta}
\end{prooftree}
$$
and allowing infinite derivation trees, called \textbf{pre-proofs}.
Clearly, these rules do not reflect the difference in nature
between $\mu$ and $\nu$. More importantly, these pre-proofs are unsound, since one can derive the empty sequent as follows:
$$
\begin{prooftree}
\Hypo{\vdots}
\Infer{1}[\rmur]{\vdash \mu X. X}
\Infer{1}[\rmur]{\vdash \mu X. X}
\Hypo{\vdots}
\Infer{1}[\rmul]{ \mu X. X\vdash}
\Infer{1}[\rmul]{ \mu X. X\vdash}
\Infer{2}[\rcut]{\vdash }
\end{prooftree}
$$
To get a proof system which is sound, we declare a pre-proof to be a proof
if and only if it satisfies the \textbf{validity condition}. This condition says, roughly speaking, that in every infinite branch there should be either a least fixed point unfolded infinitely often in the left  or a greatest fixed unfolded infinitely often in the right.

There is a natural restriction to infinitary proofs given by regular proofs, that is, proofs that have only
finitely many sub-trees. These proofs are called \textbf{circular proofs}, since they can be represented
as finite trees with loops. This restricted \textbf{cyclic proof system} is more suited for a computer
science use, as its proofs can be finitely represented and manipulated.
\end{itemize}

\bigskip
Infinitary and circular  proof systems have existed for a long time, but in the shadows, and more as technical
tools than as proper proof systems. Indeed, their infinitary nature makes them suitable intermediary objects between the syntax and the semantics, so that we usually find them in
completeness proofs. This is the case, for instance, with the
\mucalculus \textbf{refutations} of Niwinski and Walukiewicz~\citebiblio{NiwinskiW96}, which are used as an intermediary proof system in Walukiewicz's proof of completeness of the \mucalculus with respect to Kozen's
axiomatization~\citebiblio{Waluk95}.

In the last decades, infinitary proof systems have started to come out from the shadows and have begun to be considered as proper proofs (Dam and Gurov~\citebiblio{DamG02},  Dax \textit{et al.}~\citebiblio{dhl06}. Santocanale ~\citebiblio{santoCircularProofs}. Brotherston~\citebiblio{}, QuodLibet~\citebiblio{AvenhausKSW03}, $\dots$)

\medskip
Despite their compelling interest,
few proof-theoretical investigations have been done about infinitary proof theory (cut-elimination, focalization, $\dots$).




\subsection{Research hypothesis and objectives}

\subsection{National importance}

\subsection{Programme and methodology}

\subsubsection{Task A: Circular functional programs}

\begin{itemize}
\item \textbf{Long term objective:} Design typed functional programming languages for (co)-inductive data-types based on circular type systems.

  \bigskip
The use of circular proofs as type systems represents a promising new
method of writing (co)inductive programs, at least from the three following perspectives:

\begin{enumerate}
\item The programs are easier to write, more user-friendly. Indeed, there are no invariants to guess.
\item It is known that every finitary proof can be transformed into a circular one, but the converse is still an open question (cf Task B).
Circular proofs may then contain ``more proofs'' and using circular type systems may allow us to write more programs.
  \item Another interesting feature of the infinitary proofs is that their computational meaning is more explicit compared to finitary proof systems.
  \mytodo{Add an example in the background section to argument that?}
\end{enumerate}


\item \textbf{Short/Mid-term objective:} Before realizing this long term objective, one needs to ensure that the circular type sytems
  enjoy the usual fundamental properties, normalization for instance.

  A step in this direction has been done by the second author and \textit{al.},  who showed cut-elimination for circular proofs in the setting of sequent calculus~\citebiblio{}. It has been shown initially for multiplicative addivite linear-logic \MALL, but have been extended to other logics (\LL, \LK, \LJ).


  The translation between sequent calculus and
natural deduction is usually immediate in the finitary case. In the infinitary and the circular
setting, this is not the case anymore.


 Since circular typing systems  are usually in correspondence with the
formalism of natural deduction rather than with sequent calculus one,
showing their normalization is not a corrolary of the second author's result.
Thus we have for this direction two short/mid-term objectives:
  \begin{enumerate}
  \item Define properly  a validity condition for natural deduction circular proofs.
    \item Show normalization.
    \end{enumerate}

  \end{itemize}

  \mytodo{We started working on that with David Baelde, Alexis Saurin and
  Guilhem Jaber, can we explicitely cite them as collaborators for this work direction?}
\subsubsection{Task B: The expressive power of circular proofs}


The question of relating the finitary proof systems with explicit (co)induction rules and the circular proof systems is a hard question.

It has been already set as a conjecture in similar settings, for instance
by Brotherston and Simpson~\citebiblio{BrotherstonS07} in the framework of classical first-order logic with definitions.


We can ask this question at two levels: at the level of \textbf{provability}
and at the level of \textbf{computational power}.  Before discussing these two work directions,
let us set up some notations.

\begin{notation}
  Let \Sys be a proof system for a logic \textit{without} fixed points.
  We denote by \muS the finitary proof system obtained from \Sys by adding
  the explicite (co)induction rules.

  We denote by \muSi the infinitary proof system obtained from \Sys by adding
  the unfolding rules for the fixed points and considering as proofs the
  infinite derivations satisfying the validity condition.

  We denote by \muSo the circular proof system obtained by restricting
  the proofs of \muSi to the regular ones.
\end{notation}

\bigskip
For this task, we have two main research directions:
\begin{itemize}
\item \textbf{Relating the finitary and the circular proof systems at the level of provability:}
For a given proof system \Sys, does the proof systems \muS and \muSo prove the same theorems?

\medskip
In ~\citebiblio{DoumanePhD}, this question has been partially answered by showing that every provable sequent in \muS is also provable in \muSo. The converse in more difficult, and ~\citebiblio{DoumanePhD} provide only a sufficient condition on the
proof system \muS and \muSo (the invariant property) and a condition on the circular proofs to be translated (the translatability criterion) which guarantees that they can be indeed transformed into finitary proofs.

\bigskip
If the answer to this problem happens to be positive,  we see two ways to proceed:
\begin{itemize}
\item Either by finding a semantics of formulas, common to both \muS and \muSo. The idea is to show soundness for \muSo (that is, if a formula
  is provable in \muS then it is valid \wrt this semantics) and completeness for \muS (that is, if a formula is valid then it is provable in \muSo). This is the approach taken by Brotherstone and Simpson~\citebiblio{BrotherstonS07}.
\item Or by using combinatorial techniques, that is by showing a translation procedure that transforms \muSo proofs into \muS ones. This is the approach adopted in \citebiblio{DoumanePhD}. It has the advantage of being constructive by essence.
\end{itemize}


Recently, the conjecture of Brotherston and Simpson has been
 been answered by Berard and Tatsuta~\citebiblio{BerardiT17} and by Alex Simpson~\citebiblio{Simpson2017}.



\item \textbf{Relating the finitary and the circular proof systems at the level of computational power:}
For a given proof system \Sys, do the proof systems \muS and \muSo denote the same set of programs.

If the answer happens to be positive, a way to proceed is by using \textbf{proof semantics}.
For that, we have to find a domain of interpretation for proofs, common to both \muS and \muSo.
If we denote by $\interp{\pi}$, the interpretation of a \muS or a \muSo proof
$\pi$, this problem can be reformulated as follows:
For every proof $\Pi$ of \muSo (resp. \muS), is there a \muS (resp. \muSo) proof $\pi$ such that
$\interp{\Pi}=\interp{\pi}$?

An example of such proof semantics is computational ludics, which was used by the second author to compare the  \muMALLP and
\muMALLPo, the finitary and the circular extensions of polarized linear logic \MALLP.
\end{itemize}

\noindent {\bf (C):} \emph{Cyclic proof for mixed induction and
coinduction.}

Having previously developed cyclic proof systems for (program)
logics with inductive constructs, it is of clear interest to to
extend our methodology to logics featuring coinductive
structures, e.g., bisimilarity relations as widely used in
concurrency theory.  Our first step will be to formulate and
analyse a cyclic proof system for coinduction.  Since
coinductive proof often takes a circular form based on a simple
guardedness condition, it would also be interesting to compare
our proof machinery with these established forms.  Having
already considered cyclic proof for induction, we believe
dualising the concepts to coinduction should be fairly
straightforward, so the obvious next step is to consider logics
featuring mixed inductive and coinductive structures.  The more
complex soundness condition likely to be needed in the presence
of mixed induction and coinduction, taking into account the
possible nesting of least and greatest fixed points, is
suggested by the work of Niwinski and Walukiewicz on
refutations in the
$\mu$-calculus~\cite{Niwinski-Walukiewicz:97}. Having
formulated proof systems for mixed induction and coinduction,
the secondary research goals are analogous to those previously
achieved for the preceding systems for induction: the
translation of proofs featuring explicit uses of induction and
coinduction into cyclic proofs; and appropriate completeness
results for the associated infinitary systems.  (We suspect
that completeness will be non-trivial to establish, since in
the case of Niwinski and
Walukiewicz~\cite{Niwinski-Walukiewicz:97}, the set-theoretic
property of Borel determinacy is required.)  \\

\noindent {\bf (D):} \emph{Cyclic proof in other program
logics.}

Finally, it is also of interest to consider the potential for
cyclic proof in the setting of other program logics than the
frameworks considered in the strands above.  For example, one
could consider frameworks in which the syntax of the
programming language under consideration is separated from an
overlying logic, e.g.\,Plotkin's PCF language~\cite{Plotkin:77}
with LCF as the logic, or the pure-functional fragment of ML
with an appropriate logic. In such settings, cyclicity together
with an appropriate guardedness condition would presumably
replace or complement the usual structural induction principles
for the inductive datatypes of the language.

One could also investigate frameworks in which the programming
language and logic are combined in a dependent type theory.
Cyclic proof within such a framework would very likely take the
form of a construct for general recursion with appropriate
restrictions on its application, which would replace the usual
fixed recursion constructors for inductive types within the
theory.  Correctness would be ensured by proving a restricted
normalization result sufficient to ensure logical consistency,
and also by proving that closed terms of inductive type reduce
to canonical form (i.e.\,the type theory correctly models
datatypes).






\begingroup
\footnotesize
\bibliographystylebiblio{abbrv}
\bibliographybiblio{biblio}
\endgroup

\end{document}

